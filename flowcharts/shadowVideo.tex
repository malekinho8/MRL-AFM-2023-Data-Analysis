\usepackage{tikz}
\usetikzlibrary{shadows,calc}

% Parameters for the shadow
\def\shadowshift{3pt,-3pt}
\def\shadowradius{6pt}
\colorlet{innercolor}{black!10}
\colorlet{outercolor}{gray!0}

% Declare a shadow layer
\pgfdeclarelayer{shadow}
\pgfsetlayers{shadow,main}

% Shadow draw command from code 2
\newcommand\drawshadow[1]{
    \begin{pgfonlayer}{shadow}
        \shade[outercolor,inner color=innercolor,outer color=outercolor] ($(#1.south west)+(\shadowshift)+(\shadowradius/2,\shadowradius/2)$) circle (\shadowradius);
        \shade[outercolor,inner color=innercolor,outer color=outercolor] ($(#1.north west)+(\shadowshift)+(\shadowradius/2,-\shadowradius/2)$) circle (\shadowradius);
        \shade[outercolor,inner color=innercolor,outer color=outercolor] ($(#1.south east)+(\shadowshift)+(-\shadowradius/2,\shadowradius/2)$) circle (\shadowradius);
        \shade[outercolor,inner color=innercolor,outer color=outercolor] ($(#1.north east)+(\shadowshift)+(-\shadowradius/2,-\shadowradius/2)$) circle (\shadowradius);
        \shade[top color=innercolor,bottom color=outercolor] ($(#1.south west)+(\shadowshift)+(\shadowradius/2,-\shadowradius/2)$) rectangle ($(#1.south east)+(\shadowshift)+(-\shadowradius/2,\shadowradius/2)$);
        \shade[left color=innercolor,right color=outercolor] ($(#1.south east)+(\shadowshift)+(-\shadowradius/2,\shadowradius/2)$) rectangle ($(#1.north east)+(\shadowshift)+(\shadowradius/2,-\shadowradius/2)$);
        \shade[bottom color=innercolor,top color=outercolor] ($(#1.north west)+(\shadowshift)+(\shadowradius/2,-\shadowradius/2)$) rectangle ($(#1.north east)+(\shadowshift)+(-\shadowradius/2,\shadowradius/2)$);
        \shade[outercolor,right color=innercolor,left color=outercolor] ($(#1.south west)+(\shadowshift)+(-\shadowradius/2,\shadowradius/2)$) rectangle ($(#1.north west)+(\shadowshift)+(\shadowradius/2,-\shadowradius/2)$);
        \filldraw ($(#1.south west)+(\shadowshift)+(\shadowradius/2,\shadowradius/2)$) rectangle ($(#1.north east)+(\shadowshift)-(\shadowradius/2,\shadowradius/2)$);
        % \shade[outercolor,right color=black!40,left color=black!40] ($(#1.north west)+(-\shadowradius/12,\shadowradius/12)$) rectangle ($(#1.south east)+(\shadowradius/12,-\shadowradius/12)$);%Frame
    \end{pgfonlayer}
}

% create a shadow layer, so that we don't need to worry about overdrawing other things
\pgfdeclarelayer{shadow} 
\pgfsetlayers{shadow,main}

\newsavebox\mybox
\newlength\mylen

% New command shadowvideo
\newcommand\shadowvideo[2][]{
\setbox0=\hbox{\includegraphics[#1]{#2}}
\setlength\mylen{\wd0}
\ifnum\mylen<\ht0
\setlength\mylen{\ht0}
\fi
\divide \mylen by 50
\def\shadowshift{0,0}
\def\shadowradius{\the\dimexpr\mylen+\mylen+\mylen\relax}
\begin{tikzpicture}
\begin{scope}
    \clip [rounded corners=\shadowradius * 0.8] (0,0) rectangle coordinate (centerpoint) (\the\wd0, \the\ht0);
    \node[anchor=south west,inner sep=0] (image) at (0,0) {\includegraphics[#1]{#2}};
    \pic at (image.center) {Play button};
\end{scope}
\drawshadow{image}
\end{tikzpicture}
}

% Play button pic definition from code 1
\tikzset{
  Play button/.pic={
    % \fill[fill=black] (0,0) circle (0.6); % outer circle
    \fill[fill=white, fill opacity=0.75, even odd rule] (-0.1,0) circle (1) (-0.5,0.5) -- (-0.5,-0.5) -- (0.5,0) -- cycle; % inner triangle
  }
}

% \begin{tikzpicture}
%   \draw[fill, even odd rule] (0,0) circle (3) [shift={(0cm,0cm)}](0,0) circle (1);
% \end{tikzpicture}
